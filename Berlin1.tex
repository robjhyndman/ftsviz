%1. Functional data in demography. 
%    - JCGS paper.
%    - Functional principal components. 
%    - robust methods
%    - boxplots, outlier detection, rainbow plots
%    - more recent outlier detection methods

% 45 minutes

\documentclass[14pt]{beamer}
\usepackage{berlin}
%\usepackage{dsfont}

\topic{1. Tools for functional time series analysis}

\graphicspath{{figs/}}
%
%
%\def\modelblock{\begin{block}{}\vspace*{-0.75cm}
%\begin{eqnarray*}
%y_t(x) &=& f_t(x) + \sigma_t(x)\varepsilon_{t,x}\\
%f_t(x) &=& \mu(x)+ \sum_{k=1}^{K} \beta_{t,k} \, \phi_k(x) + e_t(x)\\[-.2cm]
%\end{eqnarray*}
%\end{block}\vspace*{-0.2cm}}


\begin{document}

\begin{frame}[plain]{}

\maketitle

\end{frame}


\begin{frame}{Mortality rates}\vspace*{-0.1cm}

\centerline{\animategraphics[controls,buttonsize=0.3cm,width=12.3cm]{6}{"figs/frmale"}{1}{197}}

\end{frame}


\begin{frame}{Fertility rates}\vspace*{-0.1cm}

\centerline{\animategraphics[controls,buttonsize=0.3cm,width=12.3cm]{6}{"figs/ausfert"}{1}{89}}

\end{frame}
%
%
%\begin{frame}{To consider}\large
%
%
%\structure{When we have functional data:}
%\biz
%
%\item What is a median curve?
%
%\item What is an outlier?
%
%\item How to identify outliers?
%
%\item How to order curves meaningfully?
%
%\item How to display a functional boxplot?
%
%\eiz
%
%
%\end{frame}

\section{Functional time series}

\begin{frame}{Functional time series}

\begin{block}{}\vspace*{-0.6cm}
\begin{align*}
y_t(x_i) = g_{\lambda}(z_t(x_i)) &= \begin{cases}
\log[z_t(x_i)] & \text{if $\lambda=0$;}\\
\lambda^{-1}\left[z_t^\lambda(x_i) -1\right]  & \text{otherwise}.
\end{cases} \\
 &= s_t(x_i) + \sigma_t(x_i)\varepsilon_{t,i}
\end{align*}
\end{block}
\biz
\item $z_t(x_i)$ is observed data for age $x_i$ in year~$t$,\quad $i=1,\dots,N$,\quad $t=1,\dots,T$. 
\item $\lambda$ chosen so that $\varepsilon_{t,i}\sim\text{NID}(0,1)$.
\item We assume $s_t(x)$ is a smooth function of $x$.

\item We need to estimate $s_t(x)$ from the data for $x_1 < x < x_N$.

\item We want to forecast \textbf{\alert{whole curve}} $z_{t}(x)$ for $t=T+1,\dots,T+h$.
\eiz

\end{frame}

%\begin{frame}{Functional time series}
%
%Let $g_\lambda(u) = \left\{\begin{array}{ll}
%    \log(u) & \lambda = 0;\\
%    \frac{x^\lambda-1}{\lambda} & \lambda > 0.
%    \end{array}\right.$
%
%\biz
%\item \textbf{Mortality probabilities:}
%
% $y_t(x_i) = g_0(q_t(x_i))$
%where $q_t(x_i)=$ empirical probability of death at age $x_i$.
%
%\item \textbf{Fertility rates:}
%
% $y_t(x_i) = g_{0.45}(p_t(x_i))$
%where $p_t(x_i)=$ empirical fertility rate at age $x_i$.
%
%%\item \textbf{Net migration:}
%%
%% $y_t(x_i) =$ empirical net migration at age $x_i$.
%
%\eiz
%\end{frame}
%
%

\begin{frame}{\large Smoothing functional time series}

\begin{block}{}\vspace*{-0.6cm}
\begin{align*}
y_t(x_i) &= g_{\lambda}(z_t(x_i)) 
= \begin{cases}
\log(z_t(x_i)) & \text{if $\lambda=0$;}\\
\lambda^{-1}\left(z_t^\lambda(x_i) -1\right)  & \text{otherwise}.
\end{cases} \\
 &= s_t(x_i) + \sigma_t(x_i)\varepsilon_{t,i}
\end{align*}
\end{block}

\biz
%\item $z_t(x_i)$ is mortality/fertility rate for age $x_i$ in year $t$.

\item Estimate $s_t(x)$ using penalized regression spline with a large number of knots.

\item For mortality data, use $\lambda=0$\\ and constrain $s_t(x)$ to be monotonic for $x>50$.

\item For fertility data, use $\lambda=0.4$\\ and constrain $s_t(x)$ to be concave.

\item Fit is weighted with $w_t(x_i) = \sigma_t^{-2}(x_i)$.

%\item This can be done using a modification of the {\tt gam}
%function in the {\tt mgcv} package in {\bf R}.
\eiz

\end{frame}

\begin{frame}{\large Smoothing functional time series}\vspace*{-0.2cm}

\structure{Mortality}

$D_t(x_i)=$ number of deaths at age $x_i$ in year $t$.\\
$E_t(x_i)=$ total population aged $x_i$ on June 30 in year $t$.\\
$m_t(x_i) = D_t(x_i)/E_t(x_i)=$ observed mortality rate.\\
$\mu_t(x_i)=$ ``true'' mortality rate. \pause
\begin{block}{}
\centerline{$D_t(x_i) \sim \mbox{Poisson}(E_t(x_i) \mu_t(x_i))$}
\end{block}
\pause 
$\E[m_t(x_i)] = \mu_t(x)$ and $\var[m_t(x_i)] = \mu_t(x_i)E_t^{-1}(x_i)$.
\only<5>{
\begin{textblock}{7}(0.2,5.6)
\begin{align*}
\sigma^2(x_i) &= \var(\log[m_t(x_i)]) \\
& \approx \left[ \mu_t(x_i)E_t^{-1}(x_i)\right]\mu_t(x)^{-2} \\
& = \mu_t(x_i)^{-1}E_t(x_i)^{-1}\\
\end{align*}
\end{textblock}}
\only<4-5>{\begin{textblock}{4.8}(7.7,6.7)\fontsize{13}{15}\sf
\begin{block}{Taylor series approx}\vspace*{-0.6cm}
\begin{align*}
\var[g_\lambda(X)] &= \sigma_X^2[g'_{\lambda}(\mu_X)]^2 \\
\var[\log(X)] &= \sigma_X^2 / \mu_X^{2}
\end{align*}
\end{block}\end{textblock}}

\vspace*{10cm}

\end{frame}


\begin{frame}{\large Smoothing functional time series}\vspace*{-0.2cm}


\structure{Fertility}


$B_t(x_i)=$ number of births to women aged $x_i$ in year \rlap{$t$.}\\
$E_t(x_i)=$ total population aged $x_i$ on June 30 in year $t$.\\
$f_t(x_i) = B_t(x_i)/E_t(x_i)=$ observed fertility rate.\\
$\mu_t(x_i)=$ ``true'' fertility rate.\pause
\begin{block}{}
\centerline{$B_t(x_i) \sim \mbox{Pn}(E_t(x_i) \mu_t(x_i))$}
\end{block}
\pause 
$\E[f_t(x_i)] = \mu_t(x)$ and $\var[f_t(x_i)] = \mu_t(x_i)E_t^{-1}(x_i)$.
\only<5>{
\begin{textblock}{7}(0.2,5.6)
\begin{align*}
\sigma^2(x_i) &= \var(g_\lambda[f_t(x_i)]) \\
& = \left[ \mu_t(x_i)E_t^{-1}(x_i)\right]\mu_t(x)^{2\lambda-2} \\
& = \mu_t(x_i)^{2\lambda-1}E_t(x_i)^{-1}\\
& \approx f_t(x_i)^{2\lambda-1}E_t(x_i)^{-1}
\end{align*}
\end{textblock}}
\only<4-5>{\begin{textblock}{4.8}(7.7,6.7)\fontsize{13}{15}\sf
\begin{block}{Taylor series approx}\vspace*{-0.6cm}
\begin{align*}
\var[g_\lambda(X)] &= \sigma_X^2[g'_{\lambda}(\mu_X)]^2 \\
 &= \sigma_X^2 \mu_X^{2\lambda-2}
\end{align*}
\end{block}\end{textblock}}

\vspace*{10cm}
\end{frame}


\begin{frame}{\large Smoothing functional time series}

\only<1>{\fullwidth{smmortmale3}} 
\only<2>{\fullwidth{smmortmale2}}
\only<3>{\fullwidth{smmortmale1}} 

\transfade<2-3>

\end{frame}

\begin{frame}{\large Smoothing functional time series}

\only<1>{\fullwidth{mortmale}} 
\only<2>{\fullwidth{smmortmale}}
\transfade<2>

\end{frame}




\begin{frame}{\large Smoothing functional time series}

\only<1>{\fullwidth{smfert2}} 
\only<2>{\fullwidth{smfert3}}
\only<3>{\fullwidth{smfert1}} 

\transfade<2-3>

\end{frame}

\begin{frame}{\large Smoothing functional time series}

\only<1>{\fullwidth{ausfertall}} 
\only<2>{\fullwidth{smfert}}
\transfade<2>

\end{frame}


\section{Functional principal components}

%
\begin{frame}{\large Functional principal components}

\pcablock


\begin{enumerate}\itemsep=0.0cm
\item Estimate smooth functions $s_t(x)$ using weighted penalized regression splines.


\item Compute $\mu(x)$ as $\bar{s}(x)$ across years.


\item Compute $\beta_{t,k}$ and $\phi_k(x)$ using functional principal components.


\end{enumerate}
\end{frame}



\begin{frame}{$\mu(x)$}

\fullwidth{mortmalemu}

\end{frame}


\begin{frame}{$\mu(x)$}

\fullwidth{fertmu}

\end{frame}

\begin{frame}{\large Functional principal components}


\textcolor[rgb]{1.00,0.50,0.00}{(Ramsay and Silverman,
1997,2002).}
\biz
%\item $\hat \mu(x)$ is estimated as the mean of
%$\hat{f}_1(x),\dots,\hat{f}_n(x)$.

\item In FDA, each principal component is specified by a weight
function $\phi_k(x)$.

\item The PC scores for each year are given by

\centerline{$\displaystyle \beta_{k,t} = \int \phi_k(x)
\left[\hat{s}_t(x) - \bar{s}(x)\right]dx$}

\item The aim is to:
\ben
\item Find the weight function $\phi_1(x)$ that maximizes the
variance of $\beta_{1,t}$ subject to the constraint $\int
\phi^2_i(x) dx = 1$.

\item Find the weight function $\phi_2(x)$ that maximizes the
variance of $\beta_{2,t}$ such that $\int \phi^2_i(x) dx = 1$ and
$\int\phi_1(x)\phi_2(x)dx = 0$.

\item Find the weight function $\phi_3(x)$ that \dots
\een
\eiz
\end{frame}
\begin{frame}{\large Functional principal components}
\vspace*{-0.3cm}

\structure{The optimal basis functions}

Approximate $s_t(x)$ using
\begin{block}{}
$$s_t(x) = \bar{s}(x) + \sum_{k=0}^{K} \beta_{t,k} \phi_k(x) + r_t(x)$$
\end{block}\pause


The basis function $\phi_k(x)$ which minimizes
$\displaystyle\mbox{MISE} = \frac1T \sum_{t=1}^T \int r_t^2 dx$
is the $k$th principal component (computed recursively).




\end{frame}



\begin{frame}{\large Functional principal components}\fontsize{13}{15}\sf

\structure{Computationally equivalent approach}

\biz
\item Let $s_t^*(x) = s_t(x) - \bar{s}(x)$.
\item Discretize $s_t^*(x)$ on a dense grid of $q$ equally spaced points. 
\item Denote discretized $s_t^*(x)$ as $T\times q$ matrix $\bm{G}$.

\item SVD of $\bm{G} = \bm{\Phi}{\bm{\Lambda}}\bm{\Psi}'$ where $\phi_k(x)$ is $k$th column of $\bm{\Phi}$.

\item $\beta_{t,k}$ is $(t,k)$th element of $\bm{G}\bm{\Phi}$.

\item The basis functions are orthogonal.

\item This means the coefficients series are also uncorrelated
with each other. i.e.,
$\mbox{Corr}(\hat\beta_{t,i},\hat\beta_{t,j}) = 0$ for $i\ne
j$. However, $\mbox{Corr}(\hat\beta_{t,i},\hat\beta_{s,j})\ne
0$ in general for $t\ne s$ and $i\ne j$.


\eiz

\end{frame}


\begin{frame}{\large Functional principal components}\fontsize{13}{14.8}\sf
\structure{Eigenvector approach}
\biz\itemsep=0cm\parskip=0cm
\item Let $\bm{V} = (T-1)^{-1}\bm{G}'\bm{G}$ be $m\times m$ sample covariance
matrix of $\bm{G}$.

\item Let $\bm{\Phi}_K = [\bm{\phi}_1,\dots,\bm{\phi}_K]$ consist of
the first $K$ eigenvectors of $\bm{V}$ where $K\le T-1$. The
$(i,j)$th element of $\bm{\Phi}_K$ is $\phi_i(x^*_j)$.

\item Robust versions possible using robust ``covariance'' estimation.
\eiz


\vspace*{10cm}
\end{frame}



\begin{frame}{\large Functional principal components}

\only<1>{\fullwidth{malepca}}
\only<2>{\fullwidth{malepca-robust}}
\begin{textblock}{2.5}(0.5,5)
\begin{block}{}
French male mortality
\end{block}
\end{textblock}

\only<2>{\begin{textblock}{2.5}(0.5,7.5)
\begin{alertblock}{}
Robust PCA
\end{alertblock}\end{textblock}}

\end{frame}




\begin{frame}{\large Functional principal components}

\only<1>{\fullwidth{fertpca}}
\only<2>{\fullwidth{fertpca-robust}}
\begin{textblock}{2.5}(0.5,5)
\begin{block}{}
Australian fertility
\end{block}
\end{textblock}
\only<2>{\begin{textblock}{2.5}(0.5,7.5)
\begin{alertblock}{}
Robust PCA
\end{alertblock}\end{textblock}}

\end{frame}


\section{Data visualization}

\begin{frame}{French male mortality rates}

\only<1->{\fullwidth{frenchmales}}
%\only<2->{\fullwidth{frenchmales2}}
%
%\only<3->{\begin{textblock}{7.8}(4.6,6.2)
%\begin{block}{Aims}
%  \begin{enumerate}
%    \item ``Boxplots'' for functional data
%    \item Tools for detecting outliers in functional data
%  \end{enumerate}
%\end{block}
%\end{textblock}}

\only<2>{\begin{textblock}{7}(4.6,6.2)\small
\begin{block}{Rainbow plot}
\begin{itemize}
\item Order of curves indicated by rainbow colors.
\item Other orderings are possible.
\end{itemize}\end{block}\end{textblock}}
\end{frame}


\begin{frame}{Order by functional depth}

Febrero, Galeano and Gonzalez-Manteiga (2007) proposed:
\[
o_t = \int D(y_t(x))\,dx
\]
where $D(y_t(x))$ is a univariate depth measure for each $x$.
\biz
\item $o_t$ provides an ordering of curves by ``functional depth''.
\item Problem: may not detect shape outliers.
\eiz

\end{frame}


\begin{frame}{Bivariate functional depth}

\structure{Alternative:} Apply bivariate depth measures to first two PC scores.
\vspace*{0.4cm}

Plot $\beta_{t,2}$ vs $\beta_{t,1}$
\biz
\item[\ding{229}] Each point in scatterplot represents one curve.
\item[\ding{229}] Outliers show up in bivariate score space.
\item[\ding{229}] Curves can be ordered by bivariate depth.
\eiz
\end{frame}


\begin{frame}{Robust PC scores}
\only<1>{\fullwidth{splot}} \transfade<2>
\only<2>{\fullwidth{splot2}}
\only<3>{\fullwidth{robts}}
\end{frame}



\begin{frame}{French male mortality rates}

\only<1->{\fullwidth{frmortdiff}}
\end{frame}


\begin{frame}{Robust PC scores}
\only<2>{\fullwidth{splotdiff}} \transfade<2>
\only<3>{\fullwidth{splot2diff}}
\only<1>{\fullwidth{robtsdiff}}
\end{frame}

\begin{frame}{Halfspace location depth}


\structure{The halfspace depth of a point $q$:}


\textit{(Due to Hotelling, 1929; Tukey, 1975)}

%\[
%\text{depth}_S(q) = \min_{a\in\mathbb{R}^2 \setminus 0} | \{ p\in S \mid \langle a,p\rangle \ge \langle a,q\rangle\} |
%\]

\biz
\item For each closed halfspace that contains $q$, count number of observations not in halfspace. The minimum over all halfspaces is the depth of that point. 
\item The median is the point with maximum depth (not generally unique).
\item Any point outside convex hull of the data has depth zero.

\eiz

\only<2>{\begin{textblock}{9.8}(1.5,.9)
\begin{block}{}\includegraphics[width=\textwidth]{depth}
\end{block}
\end{textblock}}

\end{frame}

\begin{frame}{Ordering by halfspace depth}

\fullwidth{frenchmales2}
\end{frame}


\begin{frame}{Bivariate bagplot}\vspace*{-0.3cm}
\hspace*{0.3cm}\emph{Due to Rousseeuw, Ruts \& Tukey (Am.Stat. 1999).}
  \begin{itemize}
  \item Rank points by halfspace location depth.
  \item Display median, 50\% convex hull and outer convex  hull (with 99\% coverage if bivariate normal).
\only<2>{ \item Boundaries contain all curves inside bags.
   \item 95\% CI for median curve also shown.}
  \end{itemize}

\only<1>{\placefig{3}{3.5}{width=9.3cm}{bagdiff}}
\only<2>{\placefig{0.0}{5.2}{width=6.3cm}{bagdiff}}
\only<2>{\placefig{6.5}{5.2}{width=6.3cm}{fbagdiff}}

\vspace*{10cm}

\end{frame}

\begin{frame}{Functional bagplot}
\fullwidth{fbagdiff}

\only<2>{\begin{textblock}{7}(1.5,6)
\begin{block}{}\fontsize{11}{12}\sf
\begin{itemize}\parskip=0cm\itemsep=0cm
\item 1850, 1854: ?
\item 1870--1871: Franco-Prussian war
\item 1914--1919: WW1
\item 1939--1945: WW2
\end{itemize}\end{block}
\end{textblock}}
\end{frame}

\begin{frame}{Kernel density estimate}

\only<1>{\fullwidth{splotdiff}}
\only<2>{\fullheight{bivardensity.jpg}}
\end{frame}


\begin{frame}{Ordering by bivariate density}

\fullwidth{frenchmales3}
\end{frame}

\begin{frame}{Functional HDR boxplot}\transfade<2>\vspace*{-0.4cm}
\begin{itemize}\itemsep=0cm
\item Bivariate HDR boxplot due to Hyndman (1996).
\item Rank points by value of kernel density estimate.
\item Display mode, 50\% and (usually) 99\% highest density regions (HDRs) and mode.
\only<2>{\item Boundaries contain all curves inside HDRs.}
\end{itemize}

\only<1>{\placefig{1}{3.5}{width=9.2cm}{hdr2}}
\only<2>{\placefig{0}{5}{width=6.3cm}{hdr2}}
\only<2>{\placefig{6.5}{5}{width=6.3cm}{fhdr}}

\vspace*{10cm}
\end{frame}


\begin{frame}{Functional HDR boxplot}
\fullwidth{fhdr}
\end{frame}



\section{References}

\begin{frame}{Selected references} %\fontsize{10}{11}\sf\vspace*{-0.2cm}

\begin{itemize}
\item[{\raisebox{-1.2cm}[0cm][0cm]{\includegraphics[width=1cm]{jcgs}}}] \fullcite{HS10}
\item[{\raisebox{-1.15cm}[0cm][0cm]{\includegraphics[width=1cm]{csda}}}] \fullcite{HU07}
\item[{\raisebox{-.481cm}[0cm][0cm]{\includegraphics[width=1cm]{Rlogo}}}] \fullcite{Rdemography}
\end{itemize}

\end{frame}


\end{document}

